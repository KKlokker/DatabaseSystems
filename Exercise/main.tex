\documentclass[12pt, a4paper]{article}
\usepackage{caption}
\usepackage{graphicx}
\usepackage{svg}
\usepackage{listings}
\usepackage{siunitx}
\usepackage{hyperref}
\def\checkmark{\tikz\fill[scale=0.4](0,.35) -- (.25,0) -- (1,.7) -- (.25,.15) -- cycle;}
\usepackage{tikz-network}
\hypersetup{
    colorlinks,
    citecolor=black,
    filecolor=black,
    linkcolor=black,
    urlcolor=black
}
\usepackage{amsmath, amsfonts, amssymb, amsthm}
\renewcommand{\thesubsubsection}{\thesubsection.\alph{subsubsection}}
\title{Algorithms and datastructures\\Exercises}
\date{2022}
\author{Kristoffer Klokker}
\begin{document}
	\maketitle
	\clearpage
	\tableofcontents
	\clearpage
		\setcounter{section}{5}
		\section{Uge}
			\subsection{Indicate the following according to figure 1.}
				\begin{figure}[h!]
					\centering
					\includegraphics[width=300px]{assets/W6E1.png}
					\caption{Two relations of a banking database}
				\end{figure}
				\subsubsection{The attributes of each realtion}
					Accounts: $acctNo$, $type$, $balance$\\
					Customers: $firstName$, $lastName$, $idNo$, $account$
				\subsubsection{The tuples of each realtion}
					\begin{itemize}
						\item $12345, savings, 12000$
						\item $23456, checking, 1000$
						\item $34567, savings, 25$\\[5mm]
						\item $Robbie$, $Banks$, $901-222$, $12345$
						\item $Lena$, $Hand$, $805-333$, $12345$
						\item $Lena$, $Hand$, $805-333$, $23456$ 
					\end{itemize}
				\subsubsection{The components of one tuble of each realtion}
					$12000$\\
					$Banks$
				\subsubsection{The relation schema of each realtion}
					$Accounts(acctNo, type, balance)$\\
					$Customers(firstName, lastName, idNo,account)$
				\subsubsection{The database schema}
					$Accounts, Customers$
				\subsubsection{A suitable domain of each attribute}
					\begin{itemize}
						\item $acctNo$ - $INT$
						\item $type$ - $VARCHAR[20]$
						\item $balance$ - $INT$
						\item $firstName$ - $VARCHAR[20]$
						\item $lastName$ - $VARCHAR[20]$
						\item $idNo$ - $CHAR[7]$
						\item $account$ - $INT$
					\end{itemize}
				\subsubsection{Another equivalent way to present each relation.}
					The attributes could simply just be in a different order.
			\subsection{In a table with the following attributes which are valid example of keys}
				$$title, year, length, genre, studioName, producerC\#$$
				\begin{itemize}
					\item title, year
					\item title, year, studioName
					\item title, length
					\item length, genre, studioName, year
				\end{itemize}
			\subsection{How many ways can relation be represented if it has:}
				\subsubsection{Four attributes and five tuples}
					$4! \cdot 5! = 2880$\\
				\subsubsection{$n$ attributes and $m$ tuples}
					$n! \cdot m!$
			\subsection{Write a database schema of the following relations}
				The datasbase schema includes\\
				$Product(make, model, type)$\\
				$PC(model, speed,ram hd, price)$\\
				$Laptop(model, speed, ram, hd ,screen, price)$\\
				$Printer(model, color, type, price)$
				\subsubsection{Write a schema for $Product$}
					CREATE TABLE Product(VARCHAR[20] maker, INT model, INT type)\\
					The type is here an int where 0 is PC, 1 is laptop and 2 is printer. There is no foreign keys due to it being the lookup table for the other relations
				\subsubsection{Write a schema for $PC$}
					CREATE TABLE PC(INT model, FLOAT speed, INT ram, BOOLEAN hd, FLOAT prize, FOREIGN KEY(Products) REFERENCES Products(model))\\
					Here the model is a reference to products, speed is gigahertz of CPU
				\subsubsection{Write a schema for $Printer$}
					CREATE TABLE Printer(INT model, BOOLEAN color, VARCHAR[20] type, FLOAT price, FOREIGN KEY(Products) REFERENCES Products(model))\\
				\subsubsection{Write an alternation for Printer and delete the attribute color}
					ALTER TABKE Printer DROP color 
				\subsubsection{Add an $od$ attribute for PC, which defaults to none an otherwise can be cd or dvd}
					ALTER TABLE PC ADD VARCHAR[20] od DEFAULT 'none'
		\section{Uge}
			\subsection{Working with linear notation}
				The following exercises uses the following schema:\\
				$Product(maker, model, type)$\\
				$PC(model, speed,ram, hd, price)$\\
				$Laptop(model, speed, ram, hd ,screen, price)$\\
				$Printer(model, color, type, price)$
				\subsubsection{PC models which have speed of at least 3.00?}
					$\pi_{model}(\sigma_{speed > 3.00}(PC))$
				\subsubsection{PC manufacturers which makes PC with a hdd with at leat 100GB}
					$\pi_{maker}(Product \bowtie \sigma_{hd >= 100}(PC))$
				\subsubsection{Find model and price of all products made by manufacturer $B$}
					\begin{align*}
						man &:= \sigma_{maker = B}(Product)\\
						PCModelPrice &:= \pi_{model, price}(man \bowtie PC)\\
						LaptopModelPrice &:= \pi_{model, price}(man \bowtie Laptop)\\
						PrinterModelPrice &:= \pi_{model, price}(man \bowtie Printer)\\
						modPrice &:= PCModelPrice \cup LaptopModelPrice \cup PrinterModelPrice
					\end{align*}
				\subsubsection{Find model numbers of all color laster printers}
					$\pi_{model}(Product \bowtie \sigma_{color = 1 AND type = laser}(Printer))$
				\subsubsection{Find manufactures that sell Laptops but not PC}
					Due to algebra not including a method for group by I have answered in form of SQL queries.\\
					SELECT (SELECT maker FROM LAPTOP NATURAL JOIN Product GROUP BY maker)  -  (SELECT maker FROM PC NATURAL JOIN Product GROUP BY maker)
				\subsubsection{Find hd size which accour in two or more PC's}
					\begin{align*}
						PC = \pi_{model,hd}(PC)\\
						PC2(model2,hd) = \pi_{model,hd}(PC)\\
						hd = \pi_{hd}(\sigma_{model != model}(PC \bowtie PC2)
					\end{align*}
			\clearpage
			\subsection{In the following data, what is the result of $\pi_{speed}(PC)$ when treated as a bag and set}
				\begin{table}[h!]
				\begin{tabular}{|l|l|l|l|l|}
				\hline
				model & speed &ram &hd &price  \\\hline
				1001 &2.66 &1024 &250 &2114 \\\hline
				1002 &2.10 & 512 &250 &995   \\\hline
				1003 &1.42 & 512 & 80 &478      \\\hline
				1004 &2.80 &1024 &250 &649    \\\hline
				1005 &3.20 &512 &250 &630     \\\hline
				1006 &3.20 &1024 &320 &1049   \\\hline
				1007 &2.20 & 1024 &200 &510  \\\hline
				1008 &2.20 &2048 &250 &770  \\\hline
				1009 &2.00 &1024 &250 &650  \\\hline
				1010 &2.80 &2048 &300 &770    \\\hline
				1011 &1.86 &2048 &160 &959  \\\hline
				1012 &2.80 &1024 &160 &649    \\\hline
				1013 &3.06 &512 &80 &529     \\\hline
				\end{tabular}
				\end{table}
				\begin{minipage}[t]{0.5\textwidth}
				\begin{center}
				Bag\\
				\begin{tabular}{|l|}
				\hline
				speed\\\hline
				2.66 \\\hline
				2.10 \\\hline
				1.42 \\\hline
				2.80 \\\hline
				3.20 \\\hline
				3.20 \\\hline
				2.20 \\\hline
				2.20 \\\hline
				2.00 \\\hline
				2.80 \\\hline
				1.86 \\\hline
				2.80 \\\hline
				3.06 \\\hline
				\end{tabular}
				\end{center}
				\end{minipage}
				\begin{minipage}[t]{0.5\textwidth}
				\begin{center}
				Set\\
				\begin{tabular}{|l|}
				\hline
				speed\\\hline
				2.66 \\\hline
				2.10 \\\hline
				1.42 \\\hline
				2.80 \\\hline
				3.20 \\\hline
				2.20 \\\hline
				2.00 \\\hline
				2.80 \\\hline
				1.86 \\\hline
				3.06 \\\hline
				\end{tabular}
				\end{center}
				\end{minipage}
	\section{Week}
		\subsection{What are the expexted FD's in the following database and what key would it have}
			\begin{itemize}
				\item name
				\item Social Security number
				\item street address
				\item city
				\item state
				\item ZIP code
				\item area code
				\item phone number
			\end{itemize}
			Social Security number $\rightarrow$ name,street address, city, state, ZIP code, area code\\
			phone number $\rightarrow$ name
			key: Social security number, phone number
		\subsection{Consider the relation with schema $R(A,B,C,D)$ and FD's $AB\rightarrow C,C\rightarrow D,D\rightarrow A$}
			\subsubsection{What are all the nontrivial FD's that follow from the given FD's? You should restrict yourself to FD's with single attributes on the rigth side}
				$AB\rightarrow C$\\
				$C\rightarrow D$\\
				$D\rightarrow A$\\
				$AB\rightarrrow D$\\
				$C\rightarrow A$
			\subsubsection{What are all the keys of R}
				$AB^+=\{C,D\}$\\
				$C^+=\{A,D\}$\\
				$D^+=\{A\}$\\
				$BC^+=\{D,A\}$\\
				$BD^+=\{A,C\}$\\
				$AB,BC,DC$
			\subsubsection{What are all the superkeys for R that is not keys?}
				$BC,BD$	
		\subsection{Find BCNF violations and decompose the schema}
			\subsubsection{$R(A,B,C,D)$ with $AB\rightarrow C$, $C\rightarrow D$, and $D\rightarrow A$}
				$AB^+=\{C,D,A\}$ - violation.\\
				$R1=R-AB^++AB=R(A,B)$\\
				$R2=AB^+=R2(A,C,D)$
			\subsubsection{$R(A,B,C,D,E)$ and $AB\rightarrow C, DE\rightarrow C, B\rightarrow D$}
				Starts wit the original relation\\
				$R(A,B,C,D,E)$\\
				The table violates $AB\rihtarrow C$\\
				$AB^+=\{C,D\}$\\
				$R1=R-AB^+=R(A,B,E)$\\
				$R2=AB^++AB=R(A,B,C,D)$\\
				But $R2$ violates $B\rightarrow D$\\
				$B^+=\{D\}$\\
				$R3=R2-B^+=R(A,B,C)$\\
				$R4=B^++B=R(B,D)$\\
				Therefore the new realtions are $R1,R3,R4$, since none now violates BCNF.
				\subsection{Perform the chase method on the relations $R(A,B,C),R(B,C,D),R(A,C,E)$ using the given FD's}
					\subsubsection{$B\rightarrow E$ and $CE\rightarrow A$}
						$A,B,C,D_1,E_1$\\
						$A_2,B,C,D,E_2$\\
						$A,B_3,C,D_3,E$\\
						$E_1=E_2$ - $B\rightarrow E$\\
						$A_2=A$ - $CE_1 \rightarrow A$\\
						$E_1=E$ - $CE \rightarrow A$\\
						Therfore on line two is now\\
						$A,B,C,D,E$
					\subsubsection{$A\rightarrow D, D\rightarrow E$ and $B\rightarrow D$}
						$A,B,C,D_1,E_1$\\
						$A_2,B,C,D,E_2$\\
						$A,B_3,C,D_3,E$\\
						$D_1=D$ - $B\rightarrow D$\\
						$A_2=A$ - $A\rightarrow D$\\
						$D_3=D$ - $A\rightarrow D$\\
						$E_1=E$ - $D\rightarrow E$
				\subsection{The following exercise is on the relation $Courses(C,T,H,R,S,G)$}
					The relation has the following FD's\\
					$C\rightarrow T$\\
					$HR\rightarrow C$\\
					$HT\rightarrow R$\\
					$HS\rightarrow R$\\
					$CS\rightarrow G$
					\subsubsection{What are all the keys for $Courses$}
						The candidate keys are:\\
						$CSH^+=\{T,G,R\}$\\
						$SHR^+=\{C,T,G\}$
					\subsubsection{Verify that the given FD's are their own minimal basis}
						$C^+=\{T\}$\\
						$HR^+=\{C,T\}$\\
						$HT^+=\{R,C\}$\\
						$HS^+=\{R,C,T\}$\\
						$CS^+=\{G,T\}$\\
						As it can be seen no one of the FD's closure result in another FD, therefore making it minimal.
					\subsubsection{Make the relation into a 3NF and check if any BCNF violation accour}
						$R1(C,T)$\\
						$R2(H,R,C,T)$\\
						$R3(H,T,R,C)$\\
						$R4(H,S,R,C,T)$\\
						$R5(C,S,G,T)$\\
						$R2$ violates BCNF due to $HT\rightarrow R$, the same with $R3$ due to $HR$ FD.
		\section{Week}
			\subsection{Create a ER diagram from the given information}
				\begin{figure}[h!]
						  \centering
						  \includegraphics[width=300px]{assets/W12E1.png}
						  \caption{ER diagram describing a bank scenario}
				\end{figure}
			\subsection{Create a ER diagram from the given information}
				\begin{figure}[h!]
						  \centering
						  \includegraphics[width=300px]{assets/W12E2.png}
						  \caption{ER diagram describing a course and student enrollment}
				\end{figure}
		 	\subsection{Create a ER diagram from the given information}
				\begin{figure}[h!]
						  \centering
						  \includegraphics[width=300px]{assets/W12E3.png}
						  \caption{ER diagram describing a department and course situation with a weak key}
				\end{figure}
			\subsection{Create a relation schema from the given diagram}
				\begin{figure}[h!]
						  \centering
						  \includegraphics[width=300px]{assets/W12E4.png}
				\end{figure}
		 		Customer(SSNO primary,phone,addr,name)\\
				Flights(number primary,day primary, aircraft)\\
				Bookings(SSNo primary, number primary, day primary, row, seat)
			\subsection{Create a relation using the different methods, from the given diagram}
				\begin{figure}[h!]
						  \centering
						  \includegraphics[width=300px]{assets/W12E5.png}
				\end{figure}
		 		\subsubsection{Straigt E/R model}
					Courses(room,number primary)\\
					Depts(name primary, chair)\\
					GivenBy(number primary, name primary)\\
					LabCourses(number primary, computerAllo)
				\subsubsection{The object oriented}
					The same as Depts, GivenBy and Courses but...\\
					LavCourses(number primary, room, CcomputerAllo)
				\subsubsection{The null method}
					The same Depts and Given by but...\\
					LabCourses(number primary, room, computerAllo)
	\section{Week}
		\subseciton{Create the following functions in Java}
			The functions are used on:\\
			Product(maker, model, type)\\
			PC(model, speed, ram, hd, price)
			\subsubsection{Given speed and RAM, find PC's model number}
				\begin{lstlisting}[language=Java]
import java.sql.*;

Class.forName("org.postgresql.Driver"); //For postsql envirement
String url = "jdbc:postgresql://" + host + "[:" + port + "]/" + database;
Connection myCon = DriverManager.getConnection(URL, username, password);

public static ArrayList<String> findModel(int speed, int ram) {
	Statement stat = myCon.createStatement("SELECT model FROM PC WHERE speed =" + speed + " AND ram =" +ram);
	ResultSet res = stat.executeQuery();
	ArrayList<String> models = new ArrayList<>();
	while(res.next()) 
		models.add(res.getString(0));
	return models;
}	
\end{lstlisting}
			\subsubsection{Given a model number, remove the tuble}
				\begin{lstlisting}[language=Java]
public static void removeModel(String model) {
	Statement stat = myCon.createStatement();
	stat.executeUpdate("REMOVE FROM PC WHERE model ='" +model+"'");
	Statement stat2 = myCon.createStatement();
	stat2.executeUpdate("REMOVE FROM Products WHERE model ='" +model+"'");
	myCon.commit();
}
				\end{lstlisting}
			\subsubsection{Decrease price of PC to 100 given mode}
				\begin{lstlisting}[language=Java]
public static void updatePrice(String model) {
	Statement stat = myCon.createStatement();
	stat.executeUpdate("UPDATE PC SET price=100 WHERE model ='" +model+"'");
	myCon.commit();
}
				\end{lstlisting}
			\subsubsection{Given all info for a PC check if model number exist if not insert}
				\begin{lstlisting}[language=Java]
public static void tryInsert(String model, int speed, int ram, int hd, int price) {
	Statement stat = myCon.createStatement("SELECT model FROM PC WHERE model='"+model+"'");
	ResultSet res = stat.executeQuery();
	if(res != null) {
		System.out.println("Model already exist");
		return
	}
	Statement stat = myCon.createStatement();
	stat.executeUpdate("INSERT INTO PC VALUES('"+model+"',"+speed+","+ram+","+hd+","+price+")");
}	myCon.commit();
				\end{lstlisting}
			\subsubsection{Ask the for a price and print maker, model and speed of PC}
				\begin{lstlisting}[language=Java]
import java.util.Scanner;
Scanner sc = new Scanner(System.in);

public static void closestPrice() {
	System.out.println("Wanted price?");
	int wantedPrice = sc.nextInt();
	int closestPrice = 0;
	String closestModel;
	Statement stat = myCon.createStatement("SELECT price, model FROM PC");
	ResultSet res = stat.executeQuery();
	while(res.next()) {
		int price = res.getInt(0);
		if(Math.abs(wantedPrice - closestPrice) > Math.abs(wantedPrice - price)) {
			closestPrice = price;
			model = res.getString(1);
		}
	}
	Statement stat2 = myCon.createStatement("SELECT maker FROM Products WHERE model = '"+model+"'");
	String maker = stat2.executeQuery().getString(0);
	System.out.println("The closest PC to the price is: "+ model + " by " + maker + " for the price of " + closestPrice);
}
				\end{lstlisting}
			\subsubsection{Ask for minimum speed,RAM, hdd and screen, and print all compatible laptops plus maker}
				\begin{lstlisting}[language=Java]
public static void findLaptop() {
	System.out.println("Wanted speed, RAM, hdd size and screen size?");
	int speed = sc.nextInt();
	int ram = sc.nextInt();
	int hdd = sc.nextInt();
	int screen = sc.nextInt();
	Statement stat = myCon.createStatement("SELECT * FROM PC");
	ResultSet res = stat.executeQuery();
	while(res.next()) {
		int PCSpeed = res.getInt(1);
		int PCRam = res.getInt(2);
		int PCHd = res.getInt(3);
		int PCScreen = rest.getInt(4);
		if(PCSpeed > speed && PCRam > ram && PCHd > hdd && PCScreen > screen) {
			String model = rest.getString(0);
			Statement stat2 = myCon.createStatement("SELECT maker FROM Product WHERE model='"+model"'");
			String maker = stat2.executeQuery().getString(0);
			System.out.println("The model: "+model+" by "+maker+" has speed: "+PCSPeed+" RAM: "+PCRAM+" HDD: "+PCHd+" price: "res.getInt(5));
		}
	}
}
				\end{lstlisting}
		\subsection{Write OSN procedues for the following tasks}
			MovieStar(name, address, gender, birtdate)\\
			MovieExec(name, address, cert\#, netWorth)\\
			Studio(name, address, presC\#)
			\subsubsection{Given name of a movie studio return net worth of president}
				\beginSQL
CREATE PROCEDURE netWorth(IN studio CHAR(15), OUT worth INT)
DECLARE foundWorth INTEGER;
BEGIN
	SET worth = 0;
	set foundWorth = SELECT netWorth FROM MovieExec WHERE cert# = (SELECT presC# FROM Studio WHERE name = studio);
	SET worth = worth + foundWorth;
END;
				\end{lstlisting}
			\subsubsection{Given name and address, return 1 if movie star, 2 if executive producer, 3 if non, 4 if both}
				\beginSQL
CREATE PROCEDURE title(IN n CHAR(15),IN a CHAR(15), OUT INT)
DECLARE star INTEGER;
DECARE prod INTEGER;
BEGIN
	SET star = SELECT COUNT(name) FROM MovieStar WHERE name = n AND address = a;
	SET prod = SELECT COUNT(name) FROM MovieExec WHERE name = n AND address = a;
	IF(1 <= star AND 1<= prod)
		return 4;
	ELSEIF (1 <= star)
		return 1;
	ELSEIF (1 <= prod)
		return 2;
	ELSE
		return 3;
	END IF;
END;
				\end{lstlisting}
			
\end{document}


